\documentclass[11pt,reqno]{amsart}

\usepackage{mathtools, mathrsfs, mathdots}
\usepackage{amsmath, amsthm, amssymb, amsfonts, amscd}
\usepackage{stmaryrd}
\usepackage[mathscr]{euscript}
\usepackage{dsfont}
\usepackage{hyperref}
\usepackage{epsfig}

% \usepackage[a4paper,left=2.85cm,right=2.85cm,top=3cm,bottom=3cm,headsep=1.2cm]{geometry}

\usepackage{todonotes}
\usepackage{tikz} 
\usepackage{tikz-cd}
\usepackage{ytableau}
\usepackage{caption}

\usepackage[all]{xy}

\usepackage{latexsym}
\usepackage{palatino}
\usepackage{enumitem}

\usepackage[normalem]{ulem}
\usepackage{upgreek}
\usepackage{mathpazo}

\usepackage{cleveref}
\usepackage{refcheck}

\usepackage{graphicx,subfigure}
\usepackage{setspace}

% \spacing{1.05}

\linespread{1.2}

\oddsidemargin=0in
\evensidemargin=0in
\textwidth=6.50in             % paper is 8.50in wide

% 1in margins at top and bottom
\headheight=10pt
\headsep=10pt
\topmargin=.25in
\textheight=8in


%%%%%%%%%%%%%%%%%%%%%%%%%%%%%%%%%%%%%%%%%%%%%%%%%%%%%%%%%%%%%%%%%%%%%%%%%%
%%%%%%%%%%%%%%%%%%%%%%%%%%%%%%%%%%%%%%%%%%%%%%%%%%%%%%%%%%%%%%%%%%%%%%%%%%
%%%%%%%%%%%%%%%%%%%%%%%%%%%%%%%%%%%%%%%%%%%%%%%%%%%%%%%%%%%%%%%%%%%%%%%%%%


\newtheorem{thm}{Theorem}[section]
\newtheorem{prop}[thm]{Proposition}
\newtheorem{lem}[thm]{Lemma}
\newtheorem{cor}[thm]{Corollary}
\newtheorem{conj}[thm]{Conjecture}
\newtheorem{prob}[thm]{Problem}
\newtheorem{ques}[thm]{Question}
\theoremstyle{definition}
\newtheorem{defn}[thm]{Definition}
\newtheorem{exam}[thm]{Example}
%% \theoremstyle{remark}
\newtheorem{rmk}[thm]{Remark}
\newtheorem{obsv}[thm]{Observation}
\theoremstyle{remark}
\newtheorem*{notation}{Notations}

\numberwithin{equation}{section}

\renewcommand\emptyset{\varnothing}

%Blackboard bold letters.
\renewcommand{\AA}{\mathbb{A}}
\newcommand{\BB}{\mathbb{B}}
\newcommand{\CC}{\mathbb{C}}
\newcommand{\DD}{\mathbb{D}}
\newcommand{\EE}{\mathbb{E}}
\newcommand{\FF}{\mathbb{F}}
\newcommand{\GG}{\mathbb{G}}
\newcommand{\HH}{\mathbb{H}}
\newcommand{\II}{\mathbb{I}}
\newcommand{\JJ}{\mathbb{J}}
\newcommand{\KK}{\mathbb{K}}
\newcommand{\LL}{\mathbb{L}}
\newcommand{\MM}{\mathbb{M}}
\newcommand{\NN}{\mathbb{N}}
\newcommand{\OO}{\mathbb{O}}
\newcommand{\PP}{\mathbb{P}}
\newcommand{\QQ}{\mathbb{Q}}
\newcommand{\RR}{\mathbb{R}}
\renewcommand{\SS}{\mathbb{S}}
\newcommand{\TT}{\mathbb{T}}
\newcommand{\UU}{\mathbb{U}}
\newcommand{\VV}{\mathbb{V}}
\newcommand{\WW}{\mathbb{W}}
\newcommand{\XX}{\mathbb{X}}
\newcommand{\YY}{\mathbb{Y}}
\newcommand{\ZZ}{\mathbb{Z}}

%Calligraphic letters
\newcommand{\cA}{\mathcal{A}}
\newcommand{\cB}{\mathcal{B}}
\newcommand{\cC}{\mathcal{C}}
\newcommand{\cD}{\mathcal{D}}
\newcommand{\cE}{\mathcal{E}}
\newcommand{\cF}{\mathcal{F}}
\newcommand{\cG}{\mathcal{G}}
\newcommand{\cH}{\mathcal{H}}
\newcommand{\cI}{\mathcal{I}}
\newcommand{\cJ}{\mathcal{J}}
\newcommand{\cK}{\mathcal{K}}
\newcommand{\cL}{\mathcal{L}}
\newcommand{\cM}{\mathcal{M}}
\newcommand{\cN}{\mathcal{N}}
\newcommand{\cO}{\mathcal{O}}
\newcommand{\cP}{\mathcal{P}}
\newcommand{\cQ}{\mathcal{Q}}
\newcommand{\cR}{\mathcal{R}}
\newcommand{\cS}{\mathcal{S}}
\newcommand{\cT}{\mathcal{T}}
\newcommand{\cU}{\mathcal{U}}
\newcommand{\cV}{\mathcal{V}}
\newcommand{\cW}{\mathcal{W}}
\newcommand{\cX}{\mathcal{X}}
\newcommand{\cY}{\mathcal{Y}}
\newcommand{\cZ}{\mathcal{Z}}

%Fraktur letters
\newcommand{\fb}{\mathfrak{b}}
\newcommand{\fg}{\mathfrak{g}}
\newcommand{\fh}{\mathfrak{h}}
\newcommand{\fm}{\mathfrak{m}}
\newcommand{\fn}{\mathfrak{n}}
\newcommand{\ft}{\mathfrak{t}}
\newcommand{\fP}{\mathfrak{P}}
\newcommand{\fQ}{\mathfrak{Q}}
\newcommand{\fS}{\mathfrak{S}}

%tilde letters
\newcommand{\tS}{\widetilde{S}}
\newcommand{\ts}{\widetilde{s}}
\newcommand{\tA}{\widetilde{A}}
\newcommand{\tG}{\widetilde{G}}
\newcommand{\tR}{\widetilde{R}}
\newcommand{\tPio}{\widetilde{\Pio}}

%bold letters
\newcommand\xx{\mathbf{x}}
\newcommand\yy{\mathbf{y}}
\newcommand\uu{\mathbf{u}}
\newcommand\mm{\mathbf{m}}

%sf letters
\newcommand\ww{\mathsf{w}}
\newcommand\vv{\mathsf{v}}
\newcommand\zz{\mathsf{z}}



\newcommand\asc{\operatorname{asc}}
\newcommand\des{\operatorname{des}}
\newcommand\inv{\operatorname{inv}}
\newcommand\inc{\operatorname{inc}}
\newcommand\sgn{\operatorname{sgn}}
\newcommand\sh{\operatorname{sh}}
\newcommand\row{\operatorname{row}}
\newcommand\col{\operatorname{col}}
\newcommand\ch{\operatorname{ch}}
\newcommand\wt{\operatorname{wt}}

\newcommand\Des{\operatorname{Des}}
\newcommand\SYT{\operatorname{SYT}}
\newcommand\SSYT{\operatorname{SSYT}}
\newcommand\Par{\operatorname{Par}}
\newcommand\Comp{\operatorname{Comp}}

\newcommand\NSym{\mathsf{NSym}}
\newcommand\Sym{\mathsf{Sym}}
\newcommand\QSym{\mathsf{QSym}}

\newcommand\gge{\succcurlyeq}
\newcommand\lle{\preccurlyeq}
\newcommand\trigeq{\unrhd}
\newcommand\trileq{\unlhd}

\newcommand\PTab{\operatorname{PTab}}


\newcommand\equivclass[1]{#1/{\sim}}
\newcommand\qand{\quad\mbox{and}\quad}
\newcommand\mand{\mbox{ \& }}

\newcommand\CHK[1]{\textcolor{red}{#1}}

\title{Good-Bad criterions and counter examples}
% \author{Byung-Hak Hwang}
% \address{Address and affiliation}
% \email{byunghakhwang@gmail.com}
\date{\today}
%
\begin{document}


\maketitle
% \tableofcontents

\section{Introduction}
In this note, I record our attempts and counter examples for them.

\section*{Notation}
Let \( P \) be a natural unit interval order on \( [n] \), and we write \( P \) as a Hessenberg
function. For example, \( P=(2,3,3) \). Sometimes, we write \( h(P)_1 = 2, h(P)_2 = 3 \) and
\( h(P)_3 = 3 \).

Let \( T \) be a \( P \)-tableau of shape \( \lambda \).
For \( 1\le i\le j\le \lambda_1 \), we denote by \( T^{[i,j]} \) the \( P \)-tableau
obtained from by the \( k \)th column of \( T \) for \( k\in[i,j] \).

For a subposet \( P'\subseteq P \), we denote by \( T[P] \) the collection of cells
whose content belongs to \( P' \).

\section{Backward connectedness criterion}
We say \emph{\( T \) forms connected straight shape} if when we decompose \( P=P_1 \sqcup \dots \sqcup P_k \), each restriction \( T[P_i] \) forms a \( P \)-tableau. In other words, for each
\( i \), the numbers of cells in each column of \( T[P_i] \) decreases.
For example, let \( P=(2,3,3,5,6,7,7) \), and
\[
  T_1 = \ytableaushort{126,34,5,7} \qand T_2 = \ytableaushort{126,34,57}.
\]
Then \( T_1 \) forms connected straight shape while \( T_2 \) doesn't.

A \( P \)-tableau \( T \) of shape \( \lambda \) is \emph{backward connected}
if for each \( i=1,\dots,\lambda_1 \), each \( P \)-tableau \( T^{[i,\lambda_1]} \) forms
connected straight shape.

\begin{description}
  \item[Criterion] For a \( P \)-tableau \( T \), we claim that \( T \) is good if and only if
  \( T \) is backward connected.
  \item[Observation] This criterion unifies the characterizations of good \( P \)-tableaux of
  hook shapes and 2-column shapes. Also, this criterion is true for the extreme case \( P=(1,2,\dots,n) \).
  \item[Counterexamples 1] The smallest counterexample.
  Let \( P=(2,4,4,5,5) \) and \( \ww=\mathsf{32451} \). Then we have
  \begin{align*}
    F_{[\ww]}(\xx)
      &= s_{3,1,1}(\xx) + 2 s_{3,2}(\xx) + 6 s_{4,1}(\xx) + 7 s_{5}(\xx) \\
      &= h_{3,1,1}(\xx) + h_{3,2}(\xx) + 3 h_{4,1}(\xx) + 2 h_{5}(\xx).
  \end{align*}
  There are 3 \( P \)-tableaux of shape \( (3,1,1) \) or \( (3,2) \):
  \[
    \ytableaushort{124,3,5} \qquad \ytableaushort{124,35} \qquad \ytableaushort{123,45}.
  \]
  All of them are backward connected, but the coefficients of \( h_{3,1,1}(\xx) \) and \( h_{3,2}(\xx)
  \) are \( 1 \) and \( 1 \). Hence the second or the third one should be bad.
  \item[Counterexamples 2] Let \( P=(2,5,5,5,6,6) \) and \( \ww=\mathsf{623451} \).
  Then we have
  \begin{align*}
    F_{[\ww]}(\xx)
      &= s_{4,1,1}(\xx) + s_{4,2}(\xx) + 4 s_{5,1}(\xx) + 4 s_{6}(\xx) \\
      &= h_{4,1,1}(\xx) + 2 h_{5,1}(\xx) + h_{6}(\xx).
  \end{align*}
  Let us observe the \( P \)-tableaux:
  \[
    \ytableaushort{1245,3,6} \qquad \ytableaushort{1245,36}.
  \]
  They are backward connected, but the second one should be bad.
  \item[Counterexamples 3] Let \( P=(2, 5, 5, 5, 6, 7, 7) \) and \( \ww=\mathsf{6723451} \).
  Then we have
  \begin{align*}
    F_{[\ww]}(\xx)
      &= s_{4,2,1}(\xx) + s_{4,3}(\xx) + 2 s_{5,1,1}(\xx) + 4 s_{5,2}(\xx) + 7 s_{6,1}(\xx) + 6 s_{7}(\xx) \\
      &= h_{4,2,1}(\xx) + h_{5,1,1}(\xx) + h_{5,2}(\xx) + 2 h_{6,1}(\xx) + h_{7}(\xx).
  \end{align*}
  Let us see the \( P \)-tableaux:
  \[
    \ytableaushort{1245,37,6} \qquad \ytableaushort{1245,367}.
  \]
  The second one should be bad while it is backward connected.
  \item[Counterexamples 4] Let \( P=(2, 5, 5, 5, 6, 7, 7) \) and \( \ww=\mathsf{6532471} \).
  Then we have
  \begin{align*}
    F_{[\ww]}(\xx)
      &= s_{4,2,1}(\xx) + s_{4,3}(\xx) + 2 s_{5,1,1}(\xx) + 4 s_{5,2}(\xx) + 7 s_{6,1}(\xx) + 6 s_{7}(\xx) \\
      &= h_{4,2,1}(\xx) + h_{5,1,1}(\xx) + h_{5,2}(\xx) + 2 h_{6,1}(\xx) + h_{7}(\xx).
  \end{align*}
  Let us see the \( P \)-tableaux:
  \[
    \ytableaushort{1342,56,7} \qquad \ytableaushort{1342,576}.
  \]
  The second one should be bad while it is backward connected.
  \item[Counterexamples 5] Let \( P=(2, 5, 5, 5, 6, 7, 8, 8) \) and \( \ww=\mathsf{53421678} \).
  Then we have
  \begin{align*}
    F_{[\ww]}(\xx)
      &= s_{4,2,1,1}(\xx) + s_{4,2,2}(\xx) + 5 s_{4,3,1}(\xx) + 5 s_{4,4}(\xx) + s_{5,1,1,1}(\xx) + 12 s_{5,2,1}(\xx)\\
        &\qquad + 21 s_{5,3}(\xx) + 13 s_{6,1,1}(\xx) + 34 s_{6,2}(\xx) + 40 s_{7,1}(\xx) + 31 s_{8}(\xx) \\
      &= h_{4,2,1,1}(\xx) + 3 h_{4,3,1}(\xx) + h_{4,4}(\xx) + 5 h_{5,2,1}(\xx) \\
        &\qquad + 6 h_{5,3}(\xx) + 2 h_{6,1,1}(\xx) + 5 h_{6,2}(\xx) + 5 h_{7,1}(\xx) + 3 h_{8}(\xx).
  \end{align*}
  Let us see the \( P \)-tableaux:
  \[
    \ytableaushort{1542,37,6,8} \qquad \ytableaushort{1342,56,78}.
  \]
  The second one should be bad while it is backward connected.
  \item[Overall] By SAGE, the number of backward connected \( P \)-tableaux are always greater than
  or equal to the number of good \( P \)-tableaux in the case \( n\le 8 \). Then I think that
  the backward connectedness criterion is a great necessary condition for goodness.
\end{description}

\section{Flippability criterion}
Since \( P \)-tableaux give the Schur expansion of the chromatic quasisymmetric function, and 
the transition matrix between \( \{s_\lambda(\xx)\} \) and \( \{h_\lambda(\xx)\} \) is
unitriangular with respect to the dominant order, we believe that good \( P \)-tableaux are
`compact' in some sense. So if there is a flippable sequence in adjacent columns,
we'd like to flip it to make the tableau more compact.

Let \( T \) be a \( P \)-tableau. A \emph{flippable ladder} is a odd-length sequence
\( (a_1,\dots,a_{2k+1}) \) of entries in \( T \) satisfying the following conditions:
\begin{enumerate}[label=(\roman*)]
  \item \( a_i \) and \( a_{i+1} \) are incomparable in \( P \) for all \( i \);
  \item \( a_i \le_P a_{i+2} \) for all \( i \); and
  \item they are concentrated in two adjacent columns.
\end{enumerate}
For \( k=1,\dots,\lambda_1 \), let \( C_k \) denote the \( k \)th column of \( T \).
It is easy to check that if a flippable ladder \( L \) lies on \( C_k \) and \( C_{k+1} \) for some
\( k \), then \( L\cap C_k = \{a_1,a_3,\dots,a_{2k+1}\} \) and \( L\cap C_{k+1} =
\{a_2,a_4,\dots,a_{2k}\} \), or vice versa. Furthermore, \( L\cap C_k \) and \( L\cap C_{k+1} \)
appear consecutively in \( C_k \) and \( C_{k+1} \) respectively.
A flippable ladder \( L \) is \emph{of type \( M \)} if \( L \cap C_k = \{a_1,a_3,\dots,a_{2k+1}\} \),
and \emph{of type \( W \)} otherwise.

Let \( L \) be a flippable ladder of type \( W \) lying on \( C_k \) and \( C_{k+1} \).



A \( P \)-tableau \( T \) is \emph{ladder-compact} if \( T \) contains no maximal flippable ladder
of type \( W \).

\begin{description}
  \item[Criterion] For a \( P \)-tableau \( T \), we claim that \( T \) is good if and only if
  \( T \) is ladder-compact.
  \item[Observation] This criterion generalize the characterization of good \( P \)-tableaux of
  2-colum shape. Also, this criterion is true for the extreme case \( P=(1,2,\dots,n) \).
  \item[Counterexamples 0] Let \( P=(2,3,3) \). Then we already have that the \( P \)-tableau
  \[
    \ytableaushort{132}
  \]
  is good, but the criterion says the tableau is bad. To avoid this case, we modify the criterion
  somewhat artificially such that compactification (flipping a ladder) on the first row is forbidden.
  \item[Counterexamples 1] The smallest counterexample. Let \( P=(2,3,4,5,5) \) and
  \( \ww=\mathsf{34521} \). Then we have
  \begin{align*}
    F_{[\ww]}(\xx)
      &= s_{2,2,1}(\xx) + s_{3,1,1}(\xx) + 5 s_{3,2}(\xx) + 6 s_{4,1}(\xx) + 6 s_5(\xx) \\
      &= h_{2,2,1}(\xx) + 3 h_{3,2}(\xx) + h_{4,1}(\xx) + h_5(\xx).
  \end{align*}
  Let us observe \( P \)-tableaux of shape \( (3,2) \):
  \[
    \ytableaushort{124,35} \qquad \ytableaushort{215,43} \qquad \ytableaushort{213,45}
  \]
  \[
    \ytableaushort{132,45} \qquad \ytableaushort{123,54}.
  \]
  The first three tableaux are not ladder-compact, and the last two tableaux are ladder-compact.
  But there should exist 3 good \( P \)-tableaux.
  Note that the backward connectedness criterion says that the second and third tableaux in the
  first row are bad.
  \item[Counterexamples 2] 
\end{description}


\section{Refinement via the first entry of \( P \)-tableaux}
Let \( \gamma \) be an equivalence class of words under the flip equivalences.
Let \( \PTab(\gamma) \) be the set of \( P \)-tableaux whose column words belong to \( \gamma \).
For \( k=1,\dots, n \), let
\[
  F_{\gamma}^{(k)}(\xx) = \sum_{\substack{T\in\PTab(\gamma) \\ T(1,1)=k}} s_{\lambda(T)}(\xx).
\]
We verified that for each \( \gamma \) and \( k \), \( F_\gamma^{(k)}(\xx) \) is \( h \)-positive
up to \( n=8 \).
Furthermore, the forward connectivity criterion
(\( \mathtt{check\_inductive\_disconnectedness\_criterion\_forward} \)) plays a valid filter
for \( F_\gamma^{(k)}(\xx) \).
In contrast, other criterions are not valid, e.g.,
\( \mathtt{check\_inductive\_disconnectedness\_criterion} \),
\( \mathtt{check\_all\_row\_connected(P, word, 'F')} \), and
\( \mathtt{check\_all\_row\_connected(P, word, 'B')} \).
The following is a counterexample for \( \mathtt{check\_all\_row\_connected(P, word, 'F')} \):

Let \( P= (2, 4, 4, 6, 6, 8, 8, 8) \), \( \ww = \mathsf{12378654} \) and \( \gamma=[\ww] \).
Then the following is the list of \( P \)-tableaux of shape \( (4,4) \) in \( \PTab(\gamma) \)
whose first entry is \( 3 \):
\begin{align*}
  T_1 = \ytableaushort{3215,7648}\qquad
  T_2 = \ytableaushort{3214,6578}\qquad
  T_3 = \ytableaushort{3214,5768}.
\end{align*}
The coefficient of \( h_{4,4}(\xx) \) in \( F_\gamma^{(k)}(\xx) \) is equal to \( 2 \).
The criterions give the following result:
\[
  \begin{tabular}{|c|c|c|c|}
    \hline
    criterion & \( T_1 \) & \( T_2 \) & \( T_3 \) \\
    \hline \hline
    \( \mathtt{check\_inductive\_disconnectedness\_criterion\_forward} \) & T & T & F \\
    \hline
    \( \mathtt{check\_inductive\_disconnectedness\_criterion} \) & F & F & F \\
    \hline
    \( \mathtt{check\_all\_row\_connected(P, word, 'F')} \) & F & T & F \\
    \hline
    \( \mathtt{check\_all\_row\_connected(P, word, 'B')} \) & F & F & F \\
    \hline
  \end{tabular}
\]
Only the first criterion gives two `True's.

Note that there are 5 counterexamples for \( \mathtt{check\_all\_row\_connected(P, word, 'F')} \)
in the case \( n=8 \). The all counterexamples are the case where the shape of incorrect tableaux
is \( (4,4) \) and \( k = 3 \).

\end{document}

